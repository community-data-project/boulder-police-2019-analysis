\documentclass[11pt]{article}
\usepackage{graphicx,amsmath,amsfonts}
\usepackage[margin=.5in]{geometry}
\usepackage{natbib}
\bibliographystyle{plainnat}

\newcommand{\bd}{\textbf}
\newcommand{\ic}{\textit}
\newcommand{\pr}{\text{P }}


\begin{document}
\section{Introduction}
\label{sec:intro}
In this report our goal is to analyze the 2018 Boulder Police Discretionary Stop Data to answer two questions:
\begin{enumerate}
\item Is police bias occurring?
\item If it is occurring, for which types of stops is it occurring most severely?
\end{enumerate}
To address these questions we will use two techniques commonly found in the literature on police bias testing: benchmark analysis and outcome testing \citep{rice_methods_2010}.

Benchmark analysis seeks to compare the demographic breakdown of the stopped or searched population against the demographics of a ``benchmark'' population, defined as all individuals whom the police might intereact with\cite{engel_comparing_2004}. Often, benchmark analyses attempt to exploit some natural problem structure to estimate the benchmarked population. Examples of a benchmark analysis include \cite{grogger_testing_2006}, which looked at the the effect of daylight on traffic stop rates by race (under the assumption that, at night, the stopping officer cannot easily determine a driver's race when deciding to make a stop). Similarly, \cite{alpert_toward_2004} propose using the demographics of the not-at-fault drivers in auto accidents to define a benchmark population of drivers against which traffic stop data can be compared.

Benchmark analysis indicates whether a racial discrepancy exists in stop or search data, but it does not explain the source of that discrepancy; it may be the case that the differences between the stopped population and the benchmark population reflect either different underlying rates of unlawful behavior, or different amounts of proximity to police. To explain the discrepency then, we turn to outcome testing. Outcome testing was originally used in the economics literature to determine whether racial bias was occurring among loan officers. By comparing the rates of loan defaulting between black and white individuals who had been granted loans, \color{red}{NAME NEEDED} determine that loan officers were granting loans to black applicants at a higher threshold than their white counterparts. In our usage, we will look at the outcomes of searches of stopped individuals. If contraband is found less often among black searchees then it indicates that officers are searching black individuals based on a lower threshold of evidence than white individuals.

Both the benchmark analysis and the outcome test can be implemented in many ways. For example, one might simply compute the stop rates of black and white individuals and compare them directly to a pre-defined benchmark population. However this approach provides no guidance as to what amount of discrepancy indicates a genuine racial difference; it gives us no way to determine statistical significance. We will therefore define more rigorous statistical models for both analysis types (benchmark and outcome) and perform each using a Bayesian framework.

The remainder of this report proceeds as follows. We will first define the models we will use for the benchmark and outcome analyses. Then we will apply the outcome test to address Question (1), ``is police bias occurring?''. Finally, we will use benchmark analysis to determine which types of stops exhibit the largest racial discrepancy.

\section{Models}
\label{sec:mods}

\subsection{Benchmark Model}
\label{subsec:bench}
The benchmark model of discretionary stops idealizes the stopping process, assuming that police perform stops entirely randomly. The stopping process is modelled as if each individual of race/ethnicity $r$ independently flips the same coin, and on a ``heads'' is subjected to a discretionary stop. From this, we will look to estimate the probability of a ``heads'' by race, denoted $\rho_r$, and then attempt to determine whether $\rho_r$ varies with $r$.

The challenge for this (and all) benchmark analyses is to determine the size of the participating population; that is, we need to determine how many individuals of each race/ethnicity are flipping a coin. By definition, the discretionary stop data only includes information on individuals whose outcome was ``heads'', which at best gives us a lower bound on the size of the benchmark population. Fortunately, the City of Boulder has done extensive work to estimate the benchmark population, outlined in the accompanying \cite{report}. We will therefore use their estimates directly, denoting the benchmark population of each race as $n_r$.

Letting $s_r$ denote the number of stops of each race/ethnicity $r$, we then have the following:
\begin{equation}
  \label{eq:bench_mod}
  \pr[s_r = k ] = \choose(n_r,k) \rho_r^k ( 1-\rho_r)^{n_r-k}
\end{equation}

Placing uniform priors on each of the $\rho_r$ yields the posterior distribution:
\begin{equation}
  \label{eq:bench_post}
  \pr[\rho_r | s_r, n_r] \propto \choose(n_r,s_r) \rho_r^{s_r} ( 1-\rho_r)^{n_r-s_r}
\end{equation}

Obsserve that Eq. (\ref{eq:bench_post}) could be rewritten as a join posterior over the multidimensional $[\rho_1,...,\rho_N]$, however since each stop is assumed to be independent then the same is true about the $\rho_r$.

To determine where racial bias is occurring most severely we will look at two metrics of discrepancy in $\rho_r$. First, we calculate the median black-white bias: the posterior median of the distribution of $B_{\text{bw}} = \rho_b - \rho_w$. When $B_{\text{bw}}$ is large and positive, it indicates a large racial disparity in stop occurrances may exist. Second, to determine the extent to which this represents a statistically significant bias rather than noise, we will use the Bayes Factor (BF).

The BF is a simple metric of compartive model performance. Given two models $M_1$ and $M_2$, each modeling the same dataset $D$, the BF is defined as the ratio:
\[
\text{BF} = \frac{\pr[M_1|D]}{\pr[M_2|D]}
\]

Applying Bayes Theorem to the top and bottom one can show:
\[
\text{BF} = \frac{ \pr[D|M_1] }{\pr[D|M_2]}
\]

Where the $\pr[D|M]$ is the normalizing, ``evidence'' term from the posterior, $\pr[D|M] = \int \pr[D,\theta|M] d \theta = \int \pr[D|\theta,M] \pr[\theta|M] d \theta$. When the BF is large, it indicates that $M_1$ is significantly more conistent with the data than model $M_2$, and vice versa when it is small.

In our case, $M_1$ will denote model (\ref{eq:post1}) while $M_2$ denotes a similar model, but where $\rho_b = \rho_w$.  

The BF can often exhibit extreme values, particularly when one model is heavily favored over another. For computational tractability we will thus work on the log scale.



\subsection{Outcome Model}
The lottery model is extremely simplified, and can only determine whether a discrepancy in stop rates exists, not if that discrepancy reflects bias on the part of the stopping officer. A slightly more sophisticated model is the ``threshold model'' \citep{simoiu_problem_2017}. The key distinction between the threshold model and the lottery model is twofold. First, the threshold model looks at the \ic{searches}, rather than discretionary stops. Second, it does not assume that the search process is random. Rather, it assumes that stopping police officers have some threshold of observable evidence, and if a stopped individual is over this threshold then the police will initiate a search.

Mathematically 
$\rho_r = \frac{1}{n_r} \sum_i \mathbb{1}(\alpha_i > \tau_{r_i}) $

Assume that the BPD have thresholds $\tau_r$ for each race/ethnicity pairing $r$, and that if individual $i$ has $\alpha_i > \tau_{r_i}$ then the BPD will initiate a discretionary stop. The ultimate question of interest here, is whether the $\tau_r$ vary across $r$. However we will not be able to answer this question in general. Instead we will attempt to answer several, more limited questions, which are indicative that the $\tau_r$ do indeed vary.

\cite{}

First, we will asses $\rho_r = P[$

\bibliography{ref.bib}


\end{document}
%%% Local Variables:
%%% mode: latex
%%% TeX-master: t
%%% End:
